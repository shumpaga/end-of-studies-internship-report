\addcontentsline{toc}{chapter}{Abstract}

\begin{abstract}

This report describes my six (6) months end-of-studies internship experience at Acute 3D -- Bentley Systems. I worked on a 3D reconstruction software called \CC and basically improved its point cloud support. One of the biggest problem faced by its users is the impossibility to reconstruct point clouds when losing scanner location information. To our knowledge, retrieving scanner locations in static point clouds is a subject never taclked, neither by the academics, nor by the industry. We describe \emph{ScanFinder}, a working method currently in an ongoing patenting process. In order to reconstruct point clouds containing multiple scanners, \CC needs to know not only scanners locations but the attribution of each point to these scanners. This enters into the realm of point cloud visibility. We describe \emph{DiskBasedVisibility}, a custom point cloud visibility algorithm also lending itself to an ongoing patent. The combination of \emph{ScanFinder} and \emph{DiskBasedVisibility} helps to reconstruct static 3D point clouds without any prior scanner location information. A second problem faced by \CC users is the difficulty to upload heavy point clouds on the cloud service for reconstruction purposes. Each time the upload fails, they need to restart from scratch. We investigate point cloud compression and present our first already integrated attempt toward a better project uploading on \CC cloud services.\\
After briefly describing the company in Chapter~\ref{ch:company} and introducing some necessary concepts in Chapter~\ref{ch:background}, we introduce \emph{ScanFinder}, \emph{DiskBasedVisibility} and point cloud compression respectively in Chapter~\ref{ch:scanfinder}, Chapter~\ref{ch:visibility} and Chapter~\ref{ch:compression}.


\end{abstract}