\chapter{Introduction}
\label{ch:introduction}

The need of 3D realistic models is increasingly present in several fields such as architecture, digital simulation or civil and structural engineering. One way to obtain a 3D model might be to build it by hand using specialized modeling softwares. In this case, the realistic aspect of the model could be doubtful. A more reliable way is to use \emph{photogrammetry} or \emph{surface reconstruction} or both at the same time.
\CC is a reality modeling software that can produce highly detailed 3D  models. It creates models of all types or scales from simple photographs of a scene to point clouds or both of them thanks to its hybrid processing. The usage of point clouds gained wide popularity because of the emergence of devices such as optical laser-based range scanners, structured light scanners, LiDAR scanners, Microsoft Kinect, etc. It is only in the past two years that \CC has started to support point clouds.\\
A scanner, whether static or mobile, outputs a point cloud. Given a point cloud $P$ assuming to lie near an unknown shape $S$, the general problem being solved is to construct a digital representation $D$ approximating $S$. A critical step during the construction of $D$ is usually the estimation and orientation of the normal of each point $p \in P$. In order to orient normals and  reconstruct point clouds acquired from static LiDAR scanners, possibly multi-scan\footnote{Point cloud made of several laser scans}, \CC needs to know the position of each scanner in the scene, on the one hand, and the attribution of each point to a scanner on the other. The scanners locations information are not always present in the metadata of some file formats, for instance LAS file format does not even provide it. Detecting the positions of multiple scanners exclusively from a point cloud is a subject neither tackled by academics nor the industry. The scanner position information is lost only due to an export with no metadata. Thus, academic research has not identified this subject as interesting. It’s only recently that scanner position information is relevant, for a few applications like \CC. The industry did not need an algorithm to retrieve the scanner position up to now and this is the main contribution presented in this internship report: a method able to detect automatically multiple scanners in a single point cloud without prior knowledge of the scene.\\
Knowing scanners position is one thing, orient normals is another. To do so, \CC needs to know for each point which scanner best sees it, there is no need to know exactly which scanner generated it. This enters into the realm of visibility of point clouds.


Yo \cite{vis1}