\chapter{Introduction}
\label{ch:introduction}

The need of 3D realistic models is increasingly present in several fields such as architecture, digital simulation or civil and structural engineering. One way to obtain a 3D model might be to build it by hand using specialized modeling softwares. In this case, the realistic aspect of the model could be doubtful. A more reliable way is to use \emph{photogrammetry} or \emph{surface reconstruction} or both at the same time. \CC is a reality modeling software that can produce highly detailed 3D  models. It creates models of all types or scales from simple photographs of a scene to point clouds or both of them thanks to its hybrid processing. The usage of point clouds gained wide popularity because of the emergence of devices such as optical laser-based range scanners, structured light scanners, LiDAR scanners, Microsoft Kinect, etc. Actually, it is only in the past two years that \CC has started to support point clouds and the common goal between each part of this internship is to improve \CC point cloud support.

The general problem being solved by \emph{surface reconstruction} is :  given a point cloud $P$ assuming to lie near an unknown shape $S$,  construct a digital representation $D$ approximating $S$. In order to reconstruct point clouds acquired from static\footnote{As opposed to mobile LiDAR scanners.} LiDAR scanners, possibly multi-scan\footnote{Point cloud made of several laser scans.}, \CC needs to know the position of each scanner in the scene, on the one hand, and the attribution of each
point to a scanner on the other. The scanners location information is not always present in point clouds metadatas, for instance LAS file format does not provide it. Moreover, some users of \CC sometimes lose this information due to a prior export with no metadata. Detecting the positions of multiple scanners exclusively from a point cloud is a subject not identified as interesting by academics, and thus not tackled since this information is almost always available from the outset. The same holds true on the industry side, it is only recently that scanner position information is relevant for a few applications like \CC. This is the main contribution presented in this internship report: a method able to detect automatically multiple scanners in a single point cloud without prior knowledge of the scene.

Knowing scanners position is one step toward supporting LAS point cloud format. \CC still needs to know for each point which scanner sees it best\footnote{There is no need to know exactly which scanner generated it.}. This enters into the realm of visibility of point clouds. One way to retrieve visibility of point clouds is to reconstruct the surface and use the underlying mesh to compute visibility. But to reconstruct the surface we need to orient the normals; a chicken-and-egg problem in our case. After some literature review on the subject, we did not find accurate method for LiDAR point clouds. Also, most papers try to find which points are visible from a precise viewpoint but in our case, as different scanners can see the same points, we want to know for each point which scanner best sees it. We introduce a custom point cloud visibility method that serves our purpose and works well with LiDAR point clouds, regardless of the sampling density.

ScanFinder and PointCloudVisibility are two contributions which serves mainly the same purpose: expand \CC input point cloud formats. Another improvement made in \CC is point cloud compression. \CC provides a cloud service which gives the opportunity for people not having any clusters or high-performance machine to do the job; reconstructing a large surface requires effective machines. The problem is that, point clouds can be very huge, up to one hundred (100) gibabyte and more. And if for a reason the upload fails, it restarts from scratch. Being able to divide by two point cloud sizes and then reduce uploading time is an interesting point for \CC cloud services. Point cloud compression can be adressed in two ways: geometric compression~\cite{compress1, compress2} or pure arithmetic compression regardless of the kind of file being compressed. We compared different compressor such as Brolti, LZMA (7Zip), Zip before integrating one of them into the product.

This report is organised as follows. Chapter~\ref{ch:company} present Bentley Systems, Acute3D, \CC and how the achieved work is positioned in the company's business line. After introducing in Chapter~\ref{ch:background} some useful definitions for a better understanding of the report, we describe the achieved work of Scan Finder, Point Cloud Visibility and Point Cloud Compression respectively in Chapter~\ref{ch:scanfinder}, Chapter~\ref{ch:visibility} and
Chapter~\ref{ch:compression}. Note that in each chapter, we recall the context, the issue addressed and the expected result before going into details. Finally Chapter~\ref{ch:conclusion} evaluates my contributions to \CC and assess what this experience has brought to me.
