\chapter{Conclusion}
\label{ch:conclusion}

Working for Acute 3D -- Bentley Systems from the 5th of February to the 3rd of August 2018 (6 months) provided me a wonderful learning experience and increased my knowledge on computational geometry and 3D reconstruction. This final internship also gave me a taste of what it is to work for a company as my previous and only one meaningful, long, internship, was in a research laboratory. The most important and immediate impacts of this training internship are: two ongoing patents, one for
\emph{ScanFinder} and one for \emph{DiskBasedVisibility} as well as my hiring as a sofware engineer for Bentley Systems.

Although it was a research internship, I noticed that working in a company has its specificities. First of all, time matters! I sometimes found myself leaving a not fully-explored lead for another one because its allowed time has been reached and another solution seems more interesting. For instance, during the research for a scan finder algorithm, our first approach described in Section~\ref{sc:grid-pattern} did not work but has not been proven to be absurd. We just moved to another more
interesting approach. However, being part of a supportive and structured work environement, working on a big and successful software, finding solutions to problems encountered by hundreds of clients is very exciting.

During these 6 months, I learned new skills not only thanks to the subjects I worked on but also thanks to our reading group. We have 1 hour of reading group every Tuesday. I myself presented two papers on point cloud visibility once. Three different subjects have been adressed during this internship: Chapter~\ref{ch:scanfinder}, Chapter~\ref{ch:visibility} and Chapter~\ref{ch:compression}. I currently have a better understanding of each topic, the previous publications,
approaches. Something specific that I learned is linear and non-linear solving. Before the internship, I knew what linear solving is but never had the oppotunity to use Least-Square or RANSAC methods. Today I understand both methods. I am able to distinguish linear from non-linear problems and even identify problems that can be solved using them.
