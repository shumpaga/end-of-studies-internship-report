\chapter{Conclusion}
\label{ch:conclusion}

Working for Acute 3D -- Bentley Systems from the 5th of February to the 3rd of August 2018 (6 months) provided me a wonderful learning experience and increased my knowledge on computational geometry and 3D reconstruction. This final internship also gave me a taste of what it is to work for a company as my previous and only one meaningful, long, internship, was in a research laboratory. The most important and immediate impacts of this training internship are: two ongoing patents, one for
\emph{ScanFinder} and one for \emph{DiskBasedVisibility} as well as my hiring as a sofware engineer for Bentley Systems.

Three different subjects have been adressed during this internship: scanner's location approximation in point cloud -- Chapter~\ref{ch:scanfinder}, visibility in point clouds -- Chapter~\ref{ch:visibility} and point cloud compression -- Chapter~\ref{ch:compression}. The common goal of all the work achieved is to improve \CC point cloud support. From the next \CC release, users will be able to benefit of point cloud compression when submitting reconstruction jobs to the cloud service. In a near future, after \emph{ScanFinder} and \emph{DiskBasedVisibility} prototypes will be validated and fully optimized, users will be able to use LAS file formats and in general static point clouds without scanner location information. This is a significant improvement because currently, there is nothing we can do for clients losing scanner information of multiscan point clouds.

Although it was a research internship, I noticed that working in a company has its specificities. First of all, time matters! I sometimes found myself leaving a not fully-explored lead for another one because its allowed time has been reached and another solution seems more interesting. For instance, during the research for a scan finder algorithm, our first approach described in Section~\ref{sc:grid-pattern} did not work but has not been proven to be absurd. We just moved to another more
interesting approach. However, being part of a supportive and structured work environment, working on a big and successful software, finding solutions to problems encountered by hundreds of clients is very exciting.

During these 6 months, I learned new skills not only thanks to the subjects I worked on but also thanks to our reading group. We have 1 hour of reading group every Tuesday. I myself presented two papers on point cloud visibility once. Three different subjects have been adressed during this internship: Chapter~\ref{ch:scanfinder}, Chapter~\ref{ch:visibility} and Chapter~\ref{ch:compression}. I currently have a better understanding of each topic, the previous publications,
approaches. Something specific that I learned is linear and non-linear solving. Before the internship, I knew what linear solving was but never had the opportunity to use Least-Square or RANSAC methods. Today I understand both methods. I am able to distinguish linear from non-linear problems and even identify problems that can be solved using them.

Improving support of static point clouds is one thing, do the same for mobile point cloud is another. As static and mobile point clouds have different functioning, this work can not be adjusted for mobile point cloud. A future work will probably be to develop scanners trajectory detection in mobile point clouds.
