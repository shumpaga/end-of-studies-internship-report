\chapter{The company}
\label{ch:company}

This chapter shed more light on \CC, the software I contributed to during these six (6) months internship as well as Bentley Systems, the company.

\section{Bentley Systems}
Bentley Systems is an American-based software development company founded by Keith A. Bentley and Barry J. Bentley in 1984.

For a bit of history\footnote{Derived from \cite{bentley}}, they introduced the commercial version of PseudoStation in 1985, which allowed users of Intergraph's VAX systems to use low-cost graphics terminals to view and modify the designs on their Intergraph IGDS (Interactive Graphics Design System) installations. Their first product was shown to potential users who were polled as to what they would be willing to pay for it. They averaged the answers, arriving at a price of \$7,943. A DOS-based version of MicroStation was introduced in 1986. Later the two other brothers joined them in the business. Today, Bentley Systems is considered to have four (4) founders: Greg Bentley (CEO), Keith A. Bentley (EVP, CTO), Barry J. Bentley, Ph.D. (EVP) and Raymond B. Bentley (EVP).

At its core, Bentley Systems is a software development company that supports the professional needs of those responsible for creating and managing the world’s infrastructure, including roadways, bridges, airports, skyscrapers, industrial and power plants as well as utility networks.  Bentley delivers solutions for the entire lifecycle of the infrastructure asset, tailored to the needs of the various professions -- the engineers, architects, planners, contractors, fabricators, IT managers,
 operators and maintenance engineers -- who will work on and work with that asset over its lifetime. Comprised of integrated applications and services built on an open platform, each solution is designed to ensure that information flows between workflow processes and project team members to enable interoperability and collaboration.

Bentley’s commitment to their user community extends beyond delivering the most complete and integrated software -- it pairs their products with exceptional service and support. Access to technical support teams 24/7, a global professional services organization and continuous learning opportunities through product training, online seminars and academic programs define their commitment to current and future generations of infrastructure professionals.

With their broad product range, strong global presence, and pronounced emphasis on their commitment to their neighbors, Bentley is much more than a software company -- they are engaged functioning members of the global community. Their successes are determined by the skills, dedication, and involvement of extraordinary Bentley colleagues around the world.

Bentley has more than 3,500 colleagues in over 50 countries, and is on track to surpass an annual revenue run rate of \$700 million. Since 2012, Bentley has invested more than \$1 billion in research, development, and acquisitions.


\section{Acute3D}
Acquired by Bentley Systems in February 2015, Acute3D is now developing \CC, as part of Bentley Systems’ Reality Modeling solutions.

Acute3D is a technological software company created in January 2011 by Jean-Philippe Pons and Renaud Keriven, by leveraging on 25 man-years of research at two major European research institutes, École des Ponts ParisTech and Centre Scientifique et Technique du Bâtiment. It won the French “most innovative startup” Awards. In 2011, Acute3D signed an industrial partnership with Autodesk, while keeping to advance its R\&D work on city-scale 3D reconstruction. In 2012, Acute3D signed industrial partnerships with other industry leaders, including Skyline Software Systems and InterAtlas. In parallel, it started to commercialize its own Smart3DCapture® (now replaced by\CC) standalone software solution, optimized for highly detailed and large-scale automatic 3D reconstruction from photographs. In 2015, Acute3D is acquired by Bentley Systems, and becomes part of their end-to-end Reality Modeling solutions. 

\section{\CC}
With \CC, you can quickly produce even the most challenging 3D models of existing conditions for infrastructure projects of all types. Without the need for expensive, specialized equipment, you can quickly create and use these highly detailed, 3D reality meshes to provide precise real-world context for design, construction, and operations decisions for use throughout the lifecycle of a project. 

Hybrid processing in ContextCapture enables the creation of engineering-ready reality meshes that incorporate the best of both worlds -- the versatility and convenience of high-resolution photography supplemented, where needed, by additional accuracy of point clouds from laser scanning.

Develop precise reality meshes affordably with less investment of time and resources in specialized acquisition devices and associated training. You can easily produce 3D models using up to 300 gigapixels of photos taken with an ordinary camera and/or 500 million points from a laser scanner, resulting in fine details, sharp edges, and geometric accuracy.

Extend your capabilities to extract value from reality modeling data with ContextCapture Editor, a 3D CAD module for editing and analyzing reality data, included with ContextCapture. ContextCapture Editor enables fast and easy manipulation of meshes of any scale as well as the generation of cross sections, extraction of ground and breaklines, and production of orthophotos, 3D PDFs, and iModels. You can integrate your meshes with GIS and engineering data to enable the intuitive search, navigation, visualization, and animation of that information within the visual context of the mesh to quickly and efficiently support the design process.  